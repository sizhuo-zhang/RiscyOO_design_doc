\subsection{Managing Speculative and Wrong-Path Instructions}\label{sec:specupdate}

Almost all instructions in the processor are speculative and may be at the wrong path, so we need to have some way to manage speculation and squash wrong-path instructions.
Since the front-end is in-order while the back-end is out-of-order, we use different schemes in the front-end and the back-end, and we explain each of the schemes next.

\subsubsection{Front-End}
The rename stage kills wrong-path instructions from the front-end (i.e., the fetch pipeline) based on the epoch in the \emph{epoch manager} (Section \mycomment{XXX}).
The fetch pipeline contains a copy of the epoch, and every fetched instruction will carry the epoch value.
At the rename stage, instructions with epochs not equal to the epoch value in the epoch manager are killed.
Whenever the PC register in the fetch pipeline needs to be redirected (because of branch mispredictions, exceptions, interrupts, or mis-speculative loads), the epoch in the epoch manager will be incremented either at the redirect time or slightly earlier.
When the PC in the fetch pipeline is truly redirected, the epoch copy in the fetch pipeline is also incremented.

Since there can be multiple consecutive redirections in the back-end, the epoch in the epoch manager should have multiple bits.
The epoch manager will recycle unused epoch values based on the epoch values of instructions seen at the rename stage.

\subsubsection{Back-End}
We use two approaches to perform squashes in the back-end:
\begin{enumerate}
    \item We delay the squash and redirection until the commit stage.
    At that time, we can simply kill every instruction in the back-end.
    \item We squash wrong-path instruction and redirect PC immediately when we find a redirection is needed (e.g., when a branch turns out to be mispredicted).
    The challenge is that we should squash only instruction younger than the instruction that triggers the redirection.
\end{enumerate}
We use the first approach for uncommon redirections, i.e., exceptions, interrupts, and the replay of mis-speculative loads that violate memory ordering.
We use the second approach for branch mispredictions only.
The implementation of these two approaches consists of the following three parts.

\noindent\textbf{Speculation tags and bit masks to track dependency on branches:}
To correctly kill wrong-path instructions in case of branch mispredictions, we assign a unique \emph{speculation tag} (type \code{SpecTag} in the source code) to each branch at the rename stage.
Every instruction will also be assigned with a speculation bit mask (type \code{SpecBits} in the source code) at the rename stage.
Each bit in the bit mask corresponds to a speculation tag.
If the bit is set, then the instruction should be killed in case the branch holding the speculation tag is mispredicted.
If a branch turns out to be predicted correctly, then the corresponding bit should be cleared from all instructions in the back-end. 
The \emph{speculation tag manager} (Section \mycomment{XXX}) assigns and recycles speculation tags, and also assigns speculation bit masks.

\noindent\textbf{Module interface to manipulate speculative instructions:}
Every module in the back-end that keeps speculative instructions should also keep the speculation bit mask for each instruction.
The module also needs to provide interface methods to allow top-level rules to squash speculative instructions or clear speculation bit masks.
All modules that keep speculative instructions except ROB provides the \code{SpeculationUpdate} interface in Figure~\ref{fig:specupdate-ifc}.
ROB provides the \code{ROB\_SpeculationUpdate} interface in Figure~\ref{fig:specupdate-ifc}.

In both interfaces, method \code{correctSpeculation} clears the speculation bit mask of each instruction in the module saccording to argument \code{mask}.
This method is called when a branch resolves to be predicted correctly.
Since multiple branches can be resolved (to be predicted correctly) in the same cycle, we use argument \code{mask} to encode the speculation tags of all these branches.

In both interfaces, method \code{incorrectSpeculation} squashes wrong-path instructions.
If argument \code{kill\_all} is true, then all instructions in the module are squashed.
This happens when commit stage commits an exception, or an interrupt, or a load that violates memory ordering.
Otherwise, only instructions whose speculation bit mask contains the speculation tag in argument \code{spec\_tag} will be squashed.
This happens when a branch is resolved and turns out to be mispredicted.
It should be noted that there can only be one source of squashing in each cycle.
In case of the ROB module, the third argument \code{inst\_tag} is the ROB index of the mispredicted branch.
It is used to simplify the squashing logic in ROB.

\begin{figure}
\begin{lstlisting}[caption={}]
interface SpeculationUpdate;
  method Action incorrectSpeculation(Bool kill_all, SpecTag kill_tag);
  method Action correctSpeculation(SpecBits mask);
endinterface
interface ROB_SpeculationUpdate;
  method Action incorrectSpeculation(Bool kill_all, SpecTag spec_tag, InstTag inst_tag);
  method Action correctSpeculation(SpecBits mask);
endinterface
\end{lstlisting}
\caption{Interface methods to manipulate speculation bit masks of modules that keep speculative instructions}\label{fig:specupdate-ifc}
\end{figure}

\noindent\textbf{Centralized entry point for manipulating speculative instructions:}
The commit stage or the branch-resolve rule does not call directly the \code{SpeculationUpdate} interface of every module to clear speculation bit masks or squashing instructions.
Instead, they both call the interface of the \emph{global speculation updater} (Section \mycomment{XXX}), which is the centralized entry point for manipulating speculative instructions.
The global speculation updater merges requests to clear speculation bit masks and then call the \code{correctSpeculation} method of every module.
It also arbitrate between requests to squash instructions and then call the \code{incorrectSpeculation} method of every module.
To some degree, having a centralized the entry point simplifies the broadcast logic for manipulating speculative instructions.

