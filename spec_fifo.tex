\subsection{Speculative FIFO}\label{sec:specfifo}

Each entry in the speculative FIFO carries a speculation bit mask (\code{SpecBits}), and the entry will be dropped if speculation is wrong.
This FIFO is often used as FIFOs between pipeline stages in the backend.

\subsubsection{Interface}

Figure~\ref{fig:specfifo-ifc} shows the interface of this module.
The interface here has been adapted slightly from what is in the source code.
Type \code{ToSpecFifo} is the content in a speculative FIFO, i.e., a payload field \code{data} and the speculation bit mask \code{spec\_bits}.
The type of the payload is parameterized by type parameter \code{t}.
Now we explain each interface method:
\begin{itemize}
    \item Methods \code{enq}, \code{deq} and \code{first}: are the common FIFO interface methods.
    It should be noted that wrong-path entries killed by mis-speculations will not show up in the dequeue port because they are already dropped.
    \item Subinterface \code{specUpdate}: See Section~\ref{sec:specupdate}.
\end{itemize}
If the module parameter \code{lazyEnq} is true, then method \code{deq} does not affect the guard of method \code{enq}, i.e., method \code{enq} cannot claim the dequeuing entry.

\begin{figure}[t]
\begin{lstlisting}[caption={}]
typedef struct {
  t data;
  SpecBits spec_bits;
} ToSpecFifo#(type t) deriving(Bits, Eq, FShow);
interface SpecFifo#(numeric type size, type t);
  method Action enq(ToSpecFifo#(t) x);
  method Action deq;
  method ToSpecFifo#(t) first;
  interface SpeculationUpdate specUpdate;
endinterface
module mkSpecFifo#(Bool lazyEnq)(SpecFifo#(size, t));
  // module implementation
endmodule
\end{lstlisting}
\caption{Interface of speclative FIFO}\label{fig:specfifo-ifc}
\end{figure}

\noindent\textbf{Conflict Matrix:}
The conflict matrix of the interface methods is:
\begin{itemize}
    \item \code{deq} $<$ \code{enq} $<$ \code{correctSpeculation}
    \item \code{incorrectSpeculation} $<$ \code{correctSpeculation}
    \item \code{incorrectSpeculation} C \code{enq}
    \item \code{deq} C \code{incorrectSpeculation}, or \code{deq} $<$ \code{incorrectSpeculation}
\end{itemize}
It should be noted that we have implemented with versions of the FIFO.
The only difference is in the ordering between method \code{deq} and  method \code{incorrectSpeculation}.

\subsubsection{Implementation}
For each FIFO entry, we use EHRs to store the valid bit, payload, and speculation bit mask.
Methods access EHRs using the appropriate ports according to the conflict matrix.
Method \code{incorrectSpeculation} will reset the valid bit of wrong-path entries, and thus leaves holes inside the FIFO.
There is an internal rule \code{canon\_deqP} which removes the head of the FIFO if the head is an invalid entry caused by mis-speculation.
Method \code{deq} and rule \code{canon\_deqP} are mutually exclusive.

If the module parameter \code{lazyEnq} (Figure~\ref{fig:specfifo-ifc}) is true, then the guard of method \code{enq} is computed using the wire which is set using port 0 of all EHRs.
This cuts off the path from method \code{deq} to \code{enq}.

\subsubsection{Future Improvement}
Although the guard of method \code{enq} can be independent from method \code{deq}, we still need to order \code{deq} $<$ \code{enq} because of double writes on the valid bits, which cannot really happen if \code{lazyEnq} is true.
This method ordering actually complicates the internal implementation of LSQ (see Section~\ref{sec:lsq})
We should be able to design a speculative FIFO with \code{deq} CF \code{enq}.

\subsubsection{Source Code}
See module \code{mkSpecFifo} in \code{//procs/procs/lib/SpecFifo.bsv}.
