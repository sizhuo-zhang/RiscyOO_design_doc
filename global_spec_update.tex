\subsection{Global Speculation Updater}\label{sec:globalspec}

The global speculation updater is the centralized point for calling the \code{SpeculationUpdate} interface of every module.

\subsubsection{Interface}
Figure~\ref{fig:globalspec-ifc} shows the interface of the module.
The module takes in two interface arguments.
Argument \code{ifc} is the aggregated interface of the \code{SpeculationUpdate} interface of every module.
That is, calling any method in \code{ifc} is effectively calls the corresponding method in the \code{SpeculationUpdate} interface of every module.
Argument \code{rob} is the \code{ROB\_SpeculationUpdate} method of ROB (Sections~\ref{sec:specupdate}).
The two arguments together contain all the interface methods to squash all the speculative instructions and clear all the speculation bit masks.
Now we explain each interface method returned by the module:
\begin{itemize}
    \item Subinterface \code{correctSpec}: provides a vector of methods to clear bits in every speculation bit mask in the back-end.
    This method is called when branch resolves to be predicted correctly.
    Having a vector of methods allows multiple branches to be resolved in one cycle.
    \item Method \code{incorrectSpec}: squashes speculative instructions by calling the\\ \code{incorrectSpeculation} method of every module in the back-end.
\end{itemize}

\begin{figure}
\begin{lstlisting}[caption={}]
interface GlobalSpecUpdate#(numeric type correctSpecPortNum, numeric type conflictWrongSpecPortNum);
  interface Vector#(correctSpecPortNum, Put#(SpecTag)) correctSpec;
  method Action incorrectSpec(Bool kill_all, SpecTag spec_tag, InstTag inst_tag);
endinterface
module mkGlobalSpecUpdate#(
  SpeculationUpdate ifc, ROB_SpeculationUpdate rob
)(GlobalSpecUpdate#(correctSpecPortNum, conflictWrongSpecPortNum));
  // module implementation
endmodule
\end{lstlisting}
\caption{Interface of global speculation updater}\label{fig:globalspec-ifc}
\end{figure}

\noindent\textbf{Conflict Matrix:}
The conflict matrix of the interface methods is:
\begin{itemize}
    \item \code{correcSpec[i]} CF \code{correctSpec[j]}
    \item \code{correctSpec} C \code{incorrectSpec}
    \item \code{incorrectSpec} C \code{incorrectSpec}
\end{itemize}
Note that method \code{incorrectSpec} is conflict with itself, so there can only be one rule calling it to squash instructions in each cycle.

There is no very specific reason for making \code{correctSpec} conflict with \code{incorrectSpec}.
Making conflicting may reduce the amount of data being broadcast, otherwise we need to broadcast \code{SpecBits} for both \code{correctSpec} and \code{incorrectSpec}.
Making them conflict might also simplify overall scheduling.

\subsubsection{Implementation}\label{sec:globalspec:impl}
The \code{incorrectSpec} method simply calls the \code{incorrectSpeculation} method in module argument \code{ifc} and \code{rob} (in Figure~\ref{fig:globalspec-ifc}).
The \code{correctSpec} methods use wires to record their input arguments, i.e., the speculation tag to be freed.
Then, a canonicalize rule  merges all the speculation tags that are freed into a speculation bit mask and calls the \code{correctSpeculation} method in module argument \code{ifc} and \code{rob}.
We manually create conflict between \code{correctSpec} and \code{incorrectSpec}.

Currently, the \code{incorrectSpec} and \code{correctSpec} methods will broadcast to all modules in a single cycle.
This may complicate routing.
To reduce the pressure on routing, we can broadcast these methods to modules in multiple cycles (e.g., via a register pipeline).
However, we still need to make sure that the \code{correctSpeculation} or \code{incorrectSpeculation} method of all modules are called atomically in one rule.
Futhermore, when a \code{incorrectSpec} method call is being broadcast to all the modules, we should block any future calls to this module until the broadcast is done.
This is because the instruction that initiate the future call may be killed by the broadcast.

\subsubsection{Source Code}
See module \code{mkGlobalSpecUpdate} in \code{//procs/lib/GlobalSpecUpdate.bsv}.

\subsubsection{Future Improvement}
We should pipeline the broadcast as described in Section~\ref{sec:globalspec:impl}.
