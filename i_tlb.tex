\subsection{L1 I TLB}

The L1 I TLB is similar to the D TLB except that it will block in case of a miss (i.e., no hit-under-miss or multiple outstanding misses).
As a consequence, it also sends responses in the order of requests.

\subsubsection{Interface}

\begin{figure}
\begin{lstlisting}[caption={}]
interface ITlb;
  method Bool flush_done;
  method Action flush;
  method Action updateVMInfo(VMInfo vm);
  method Bool noPendingReq;
  interface Server#(Addr, TlbResp) to_proc;
  interface ITlbToParent toParent;
  interface Perf#(L1TlbPerfType) perf;
endinterface
\end{lstlisting}
\caption{Interface of I TLB}\label{fig:i-tlb-ifc}
\end{figure}

Figure~\ref{fig:i-tlb-ifc} shows the interface of the I TLB:
\begin{itemize}
    \item Subinterface \code{to\_proc}: contains a request and response interfaces.
    The request interface takes in a virtual address, and the response interface returns the translation result.
    The response order matches the request order.
    \item Other methods are similar to those in D TLB (Section~\ref{sec:d-tlb})
\end{itemize}

\subsubsection{Implementation}
The implementation uses one register to a missing request, so it will block if there is a miss.
The detailed implementation is a simplified version of the D TLB (Section~\ref{sec:d-tlb}).

\subsubsection{Future Improvement}
We can try to make the I TLB able to support multiple misses.
However, there may not be any improvement because of the locality in instruction fetch.

\subsubsection{Source Code}
See module \code{mkITlb} in \code{//procs/lib/ITlb.bsv}.
