\subsection{CSR Register File}
The CSR register file contains all the CSRs.
Besides the official CSRs in the RISC-V ISA, it also includes two custom CSRs:
\begin{itemize}
    \item The stats CSR: controls whether performance counters will be incremented or not.
    When the stats CSR of one processor core is written, the write will be automatically propagated to all other cores.
    \item The terminate CSR: will terminate the whole processor when any value is written to it.
\end{itemize}

\subsubsection{Interface}

\begin{figure}
\begin{lstlisting}[caption={}]
interface CsrFile;
  method Data rd(CSR csr);
  method Action csrInstWr(CSR csr, Data x);
  method Bool fpuInstNeedWr(Bit#(5) fflags, Bool fpu_dirty);
  method Action fpuInstWr(Bit#(5) fflags);
  method Maybe#(Interrupt) pending_interrupt;
  method ActionValue#(Addr) trap(Trap t, Addr pc, Addr faultAddr);
  method ActionValue#(Addr) sret;
  method ActionValue#(Addr) mret;
  method VMInfo vmI;
  method VMInfo vmD;
  method CsrDecodeInfo decodeInfo;
  method Action incInstret(SupCnt x);
  method Action setTime(Data t);
  method Bit#(1) getMSIP;
  method Action setMSIP(Bit#(1) v);
  method Action setMTIP(Bit#(1) v);
  method Bool doPerfStats;
  method ActionValue#(Bool) sendDoStats;
  method Action recvDoStats(Bool s);
  method ActionValue#(void) terminate;
endinterface
module mkCsrFile#(Data hartid)(CsrFile);
  // implementation
endmodule
\end{lstlisting}
\caption{Interface of CSR register file}\label{fig:csr-file-ifc}
\end{figure}

Figure~\ref{fig:csr-file-ifc} shows the interface of the CSR register file:
\begin{itemize}
    \item Method \code{rd}: returns the value of a given CSR.
    \item Method \code{csrInstWr}: updates a CSR when a \inst{CSRRW} instruction is committed.
    \item Method \code{fpuInstNeedWr}: returns whether a floating-point instruction needs to update the \code{fcsr} CSR.
    \item Method \code{fpuInstWr}: updates the \code{fcsr} CSR when a a floating-point instruction is committed.
    \item Method \code{pending\_interrupt}: returns any pending interrupts.
    \item Method \code{trap}: updates the CSRs for an exception or interrupt (at commit stage), and returns the next PC (i.e., the entry to the interrupt/exception handler).
    \item Methods \code{sret} and \code{mret}: udpates the CSRs when a return-from-trap instruction (\inst{SRET} or \inst{MRET}) is committed, and returns the next PC.
    \item Methods \code{vmI}, \code{vmD} and \code{decodeInfo}: return some CSR contents that may be used in other parts of the processor.
    \item Method \code{incInstret}: increments the \code{minstret} CSR by the number of committed instructions in this cycle.
    \item Method \code{setTime}: updates the local copy of the machine time (the original machine time register is maintained in the uncore).
    \item Methods \code{getMSIP}, \code{setMSIP}, \code{setMTIP}: provide accesses on the interrupt pending bits (such accesses are made by the uncore).
    \item Method \code{doPerfStats}: returns whether performance counters should be incremented or not.
    \item Methods \code{sendDoStats} and \code{recvDoStats}: propagate the changes on the stats CSR between processor cores.
    \item Method \code{terminate}: indicates the whole processor should be shutdown because of writes on the terminate CSR, if the method is ready.
\end{itemize}
The module argument \code{hartid} is the core ID.

\subsubsection{Implementation}
Since there is no read-after-write hazards on CSRs, we use Bluespec \code{mkConfigReg} to implement each CSR.
This avoids unncessary constraints on rule scheduling.

\subsubsection{Source Code}
See module \code{mkCsrFile} in \code{//procs/lib/CsrFile.bsv}.
