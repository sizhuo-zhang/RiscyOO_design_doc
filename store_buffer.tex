\subsection{Store Buffer (SB)}\label{sec:sb}

The store buffer (SB) holds committed stores retired from the LSQ.
It can  merge stores for the same cache line and issue multiple requests for different cache lines to the L1 cache.
After a SB entry is issued to the memory system, the entry still stays in SB to merge future stores for the same cache line.
When the L1 cache has retrieved the exclusive permission for the cache line, and L1 cache is updated.
SB is only used in the WMM implementation.

\subsubsection{Interface}

\begin{figure}
\begin{lstlisting}[caption={}]
interface StoreBuffer;
  method Bool isEmpty;
  method Maybe#(SBIndex) getEnqIndex(Addr paddr);
  method Action enq(SBIndex idx, Addr paddr, ByteEn be, Data data);
  method ActionValue#(SBEntry) deq(SBIndex idx);
  method ActionValue#(Tuple2#(SBIndex, SBEntry)) issue;
  method SBSearchRes search(Addr paddr, ByteEn be);
  method Bool noMatchLdQ(Addr paddr, ByteEn be); 
  method Bool noMatchStQ(Addr paddr, ByteEn be); 
endinterface
module mkStoreBufferEhr(StoreBuffer);
  // module implementation
endmodule
\end{lstlisting}
\caption{Interface of SB}\label{fig:sb-ifc}
\end{figure}

Figure~\ref{fig:sb-ifc} shows the interface of SB.
Now we explain each interface method:
\begin{itemize}
    \item Method \code{isEmpty}: returns true if SB is empty.
    This is called typically when a fence instruction is executed for a system instruction is committed.
    
    \item Method \code{getEnqIndex}: returns the index for a SB entry to insert a new store.
    The entry may be a new entry or an existing entry which contains some other stores for the same cache line.
    An \code{Invalid} is returned if no entry can be found.
    
    \item Method \code{enq}: inserts a store into the SB entry at index \code{idx}.
    Argument \code{idx} should be the return value of \code{getEnqIndex}.
    Arguments \code{be} and \code{data} should be shifted to represent a masked 8B aligned data.
    (Lower bits of \code{paddr} are not used.)
    
    \item Method \code{deq}: removes the SB entry at index \code{idx}.
    This is called when L1 gets exclusive permission for the cache line.
    
    \item Method \code{issue}: returns a SB entry that has not been issued to L1 before.
    The entry is not removed from SB.
    L1 will use the address in the SB entry to fetch exclusive permission for the cache line.
    
    \item Method \code{search}: finds bypass and stall for a issuing load in SB.
    Argument \code{be} should be shifted to indicate the valid bytes in a 8B aligned data.
    In case bypass is possible, a 8B aligned data is returned, so the caller needs to shift data before writing to register file.
    
    \item Methods \code{noMatchLdQ} and \code{noMatchStQ}: check if a Lr/Sc/Amo address exists in SB.
    They are called when we try to dequeue a Lr/Sc/Amo from LQ or SQ.
\end{itemize}

\noindent\textbf{Conflict Matrix:}
The conflict matrix of SB is:
\begin{itemize}
    \item \{\code{isEmpty}, \code{issue}, \code{search}\} $<$ \code{deq} $<$ \code{getEnqIndex} $<$ \code{enq}
    \item All others are CF
\end{itemize}

\subsubsection{Implementation}
The implementation uses a vector of EHRs to keep the SB entries.
We use a FIFO \code{freeQ} to keep the indices of free SB entries.
We also use a FIFO \code{issueQ} to keep the indices of SB entries that need to be issued to L1.
A newly allocated entry in method \code{enq} will be inserted into \code{issueQ}, and method \code{issue} will dequeue \code{issueQ} and issue the SB entry pointed by the head of \code{issueQ}.

\subsubsection{Source Code}
See module \code{mkStoreBufferEhr} in \code{//procs/lib/StoreBuffer.bsv}.

