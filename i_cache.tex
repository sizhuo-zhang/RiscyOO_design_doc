\subsection{L1 I Cache}

The I cache is very similar to the D cache.
The major difference is that I cache sends responses in the same order as the original requests, even though it may processes misses out of order.
The I cache is also coherent with the parent L2.

\subsubsection{Interface}

\begin{figure}
\begin{lstlisting}[caption={}]
interface ICoCache;
  interface Server#(Addr, Vector#(SupSize, Maybe#(Instruction))) to_proc;
  interface ChildCacheToParent#(L1Way, void) to_parent;
  interface Perf#(L1IPerfType) perf;
  interface Get#(L1ICRqStuck) cRqStuck;
  interface Get#(L1IPRqStuck) pRqStuck;
endinterface
module mkICoCache(ICoCache);
  // implementation
endmodule
\end{lstlisting}
\caption{Interface of I cache}\label{fig:i-cache-ifc}
\end{figure}

Figure~\ref{fig:i-cache-ifc} shows the interface of the I cache.
Now we explain the interface:
\begin{itemize}
    \item Subinterface \code{to\_proc}: is a pair of request and response interfaces.
    The request interface takes in an address.
    For this request, the I cache will fetch consecutive instructions starting from this address.
    The fetch will stop until it hits the cache boundary or the number of the fetched instruction reaches the superscalar size.
    The response order is the same as the request order, but the I cache can process multiple misses in parallel.
    \item Other interfaces are similar to those in D cache (Section~\ref{sec:d-cache}).
\end{itemize}

\subsubsection{Implementation}
The implementation is very similar to D cache except that responses are sent to the core in order.

\subsubsection{Source Code}
See the followings:
\begin{itemize}
    \item module \code{mkICoCache} in \code{//procs/lib/L1CoCache.bsv}, and
    \item module \code{mkIBank} in \code{//coherence/src/IBank.bsv}.
\end{itemize}

