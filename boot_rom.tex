\subsection{Boot Rom}
The boot rom module contains the very first instructions executed when the processor starts.
In the boot-rom program (\code{//procs/rom/rom.S}), core 0 and other cores do different things:
\begin{itemize}
    \item Core 0 does the following:
    \begin{enumerate}
        \item Use MMIO access to request memory loader to start the initialization of DRAM.
        \item Keep checking memory loader until the DRAM initialization is done.
        \item Set the MSIP bits of all other cores to wake them up.
        \item Jump to BBL.
    \end{enumerate}
    \item Core non-zero does the following:
    \begin{enumerate}
        \item Set up \code{mtvec} CSR for a customized trap handler.
        \item Turn on MSIE to receive the wake-up signal from core 0.
        \item Loop forever to wait for the wake-up from core 0.
        \item After receiving the wake-up signal from core 0, the trap handler will jump to BBL.
    \end{enumerate}
\end{itemize}
The content of the boot rom is initialized by the x86 host software before the processor starts.
The implementation of the module is straightforward, so we skip the details here.

\subsubsection{Source Code}
See module \code{mkBootRom} in \code{//procs/lib/BootRom.bsv}.
