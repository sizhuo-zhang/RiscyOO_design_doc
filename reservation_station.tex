\subsection{Reservation Station}\label{sec:rs}
Reservation station is the first module of every execution pipeline.
Rename stage enters instructions into one of the reservation stations.
The module tracks whether unissued instructions are ready to issue to execution, and schedules ready instructions to issue.

\subsubsection{Interface}\label{sec:rs:ifc}
Figure~\ref{fig:rs-ifc} shows the module interface of the reservation station.
Type \code{ToReservationStation} contains all the fields in a reservation-station entry:
\begin{itemize}
    \item Field \code{data}: is the information that an instruction wants to store in the entry.
    Its type is parameterized
    \item Field \code{regs}: is the indexes of the physical registers used by the instruction.
    \item Field \code{spec\_bits}: is speculation bit mask of the instruction.
    \item Field \code{spec\_tag}: is the speculation tag of the instruction if it is a branch.
    In future implementations, this field should be moved into the \code{data} field to save area, because not all instruction need this field.
    \item Field \code{regs\_ready}: is the register presence bits of all the source registers of the instruction.
\end{itemize}
Interface \code{ReservationStation} has three type parameters:
\begin{itemize}
    \item Numeric type \code{size}: is the number of entries in the reservation station.
    \item Numeric type \code{setRegReadyNum}: is the number of concurrent wakeups.
    \item Type \code{a}: is the type parameter for \code{ToReservationStation}.
\end{itemize}
Now we explain each interface method:
\begin{itemize}
    \item Method \code{enq}: enters an instruction into the reservation station.
    \item Method \code{canEnq}: returns the guard of method \code{enq}, i.e., whether there is a free entry.
    \item Method \code{setRobEnqTime}: tells the module about the conceptual enqueue pointer of ROB.
    This information is used in the age-based scheduling policy, i.e., older instructions have higher priority in being issued.
    \item Method \code{dispatchData}: returns the ready entry that is scheduled to be issued.
    The guard is false if there is no entry to issue.
    \item Method \code{doDispatch}: removes the ready entry that is scheduled to be issued from the module.
    \item Subinterface \code{setRegReady}: provides a vector of methods to wake up instructions, i.e., a physical register becomes ready.
    \item Method \code{approximateCount}: returns the approximate number of occupied entries in the module.
    The result is used by the rename stage as a heuristic to choose which reservation station to enter an instruction if there are multiple identical execution pipelines (e.g., there can be two ALU pipelines).
    \item Subinterface \code{specUpdate}: manipulates speculative states (Section~\ref{sec:specupdate}).
\end{itemize}

\begin{figure}[t]
\begin{lstlisting}[caption={}]
typedef struct {
  a data;
  PhyRegs regs;
  InstTag tag;
  SpecBits spec_bits;
  Maybe#(SpecTag) spec_tag;
  RegsReady regs_ready;
} ToReservationStation#(type a) deriving(Bits, Eq, FShow);
interface ReservationStation#(numeric type size, numeric type setRegReadyNum, type a);
  method Action enq(ToReservationStation#(a) x);
  method Bool canEnq;
  method Action setRobEnqTime(InstTime t);
  method ToReservationStation#(a) dispatchData;
  method Action doDispatch;
  interface Vector#(setRegReadyNum, Put#(Maybe#(PhyRIndx))) setRegReady;
  method Bit#(TLog#(TAdd#(size, 1))) approximateCount;
  interface SpeculationUpdate specUpdate;
endinterface
module mkReservationStation(ReservationStation#(size, setRegReadyNum, a));
  // module implementation
endmodule
\end{lstlisting}
\caption{Interface of reservation station}\label{fig:rs-ifc}
\end{figure}

\noindent\textbf{Conflict Matrix:}
The conflict matrix of the module is:
\begin{itemize}
    \item \code{setRegReady[0]} $<$ \code{setRegReady[1]} $<$ $\cdots$ $<$ \code{setRegReady[setRegReadyNum-1]}
    \item \{\code{setRegReady}, \code{dispatchData}, \code{doDispatch}, \code{canEnq}\} $<$ \code{enq}
    \item \code{setRobEnqTime} $<$ \code{dispatchData} $<$ \code{doDispatch}
    \item \code{dispatchData} $<$ \code{incorrectSpeculation}
    \item \code{incorrectSpeculation} C \{\code{enq}, \code{doDispatch}\}
    \item \{\code{enq}, \code{dispatchData}, \code{doDispatch}, \code{incorrectSpeculation}\} $<$ \code{correctSpeculatoin}
    \item All others are conflict free.
\end{itemize}

We make \code{dispatchData} and \code{doDispatch} conflict free with \code{setRegReady}, because \code{dispatchData} and \code{doDispatch} do not need to see the effect of \code{setRegReady} in the same cycle.
An instruction can be issued many cycles after it is waken up.
This cuts off the bypass path from \code{setRegReady} to \code{dispatchData} (i.e., selecting a ready entry).

We make \code{canEnq} conflict free with \code{doDispatch}, because the reservation station does not need to have pipelined behavior, i.e., being able to remove and insert simultaneously when it is full.
This cuts off bypass from \code{doDispatch} to \code{canEnq} which is used in the rename stage.

We order \code{enq} after \code{setRegReady}, \code{dispatchData}, \code{doDispatch}, and \code{canEnq} because \code{enq} happens at the earliest stage, i.e., the rename stage.

We order \code{setRobEnqTime} before \code{dispatchData} and \code{doDispatch} because scheduling an instruction to issues uses the ROB enqueue pointer as a heuristic.

We order \code{incorrectSpeculation} conflict with \code{enq} and \code{doDispatch} to cut off any bypass paths.

We order \code{correctSpeculation} after other methods because of the convention stated in Section~\ref{sec:specupdate}.

\subsubsection{Implementation}
The module uses a set of EHRs to store an entry, i.e., one EHR per field.
The module also use a vector of EHRs to store the valid bits of each entry.
Methods \code{setRegReady}, \code{correctSpeculation} and \code{incorrectSpeculation} access the EHRs directly using EHR ports according to the conflict matrix.

Method \code{canEnq} (i.e., the guard of \code{enq}) does not check the valid bits directly.
An internal rule searches through valid bits using EHR port 0 to find an empty entry, and then set a wire with the entry index (a \code{Maybe} type).
Method \code{canEnq} checks if the wire contains a valid index.
Method \code{enq} writes to entry indexed by the wire value, but it writes to the appropriate EHR port of data fields according to the conflict matrix.

Methods \code{dispatchData} and \code{doDispatch} does not check the register-ready bits (\code{reg\_ready}) in the EHRs directly.
The module uses a vector of wires to catch the values of the register-ready bits with EHR port 0.
The methods use the values of the wires to determine which one is ready to issue.
Wires are acceptable because both \code{dispatchData} and \code{doDispatch} are conflict-free with \code{setRegReady}.
For fields other than register-ready bits, the methods access the EHRs directly using the appropriate ports.

Method \code{setRobEnqTime} uses a wire to record its argument.
And method \code{dispatchData} uses the wire value in the issue-scheduling policy.
Wire is acceptable because this is a performance feature, unrelated to correctness.

We also manually make \code{incorrectSpeculation} conflict with \code{enq} and \code{doDispatch}.

\subsubsection{Source Code}
See the followings:
\begin{itemize}
    \item module \code{mkReservationStation} in \code{//procs/lib/ReservationStationEhr.bsv},
    \item module \code{mkReservationStationAlu} in \code{//procs/RV64G\_OOO/ReservationStationAlu.bsv},
    \item module \code{mkReservationStationMem} in \code{//procs/RV64G\_OOO/ReservationStationMem.bsv}, and
    \item module \code{mkReservationStationFpuMulDiv} in\\ \code{//procs/RV64G\_OOO/ReservationStationFpuMulDiv.bsv}.
\end{itemize}

\subsubsection{Future Implementation}
We should move field \code{spec\_tag} into field \code{data} in type \code{ToReservationStation} as described in Section~\ref{sec:rs:ifc}.
